\documentclass[12pt,a4paper,openright]{report}
\usepackage[utf8]{inputenc}
\usepackage[english]{babel}
\usepackage{amsmath}
\usepackage{amsfonts}
\usepackage{amssymb}
\usepackage{graphicx}
\usepackage[left=3cm,right=3cm,top=3cm,bottom=3cm]{geometry}
\author{A. Sanfratello, A. Beconcini, F. Mola}
\title{Tranformation from cartesian coordinates to polar coordinates using CORDIC}

\newcommand{\HRule}{\rule{\linewidth}{0.5mm}}

\begin{document}


\begin{titlepage}
\begin{center}
	\includegraphics[scale=.60]{Unipi_logo.jpg}\\[3cm]
	\textsc{\Large Digital System Design Project}
	\HRule \\[0.4cm]
{ \huge \bfseries Tranformation from cartesian coordinates to polar coordinates using CORDIC \\[0.4cm] }
	\HRule \\[4cm]
	\noindent
	\begin{minipage}{0.4\textwidth}
	\begin{flushleft} \large
	\emph{Authors:}\\
	Alessio Sanfratello\\
	Andrea Beconcini\\
	Francesco Mola 
	\end{flushleft}
	\end{minipage}%
	\begin{minipage}{0.4\textwidth}
	\begin{flushright} \large
	\emph{Supervisor:} \\
	Prof. Luca Fanucci
	\end{flushright}
	\end{minipage}

	\vfill
	{\large Academic Year 2014/2015}
\end{center}
\end{titlepage}

\tableofcontents

\chapter{Introduction}

\section{The CORDIC algorithm}

\section{Specification of requirements}




\chapter{Implementation}

\section{Architecture}

\section{Implementation using VHDL}




\chapter{Testing}

\section{Choice of appropriate testbench}

\section{Generation of testing data using C++}




\chapter{Systesis using Xilinx ISE Tool}




\chapter{Conclusions}




\appendix

\chapter{Source code}

\end{document}