\documentclass[12pt,a4paper]{report}

\usepackage[utf8]{inputenc}
\usepackage[english]{babel}
\usepackage{amsmath}
\usepackage{amsfonts}
\usepackage{amssymb}
\usepackage{graphicx}
\usepackage{listings}
\usepackage[left=3cm,right=3cm,top=3cm,bottom=3cm]{geometry}
\usepackage{url} % formattazione url
\usepackage{color}

%colori
\definecolor{lgrey}{rgb}{0.95,0.95,0.95}
\definecolor{dblue}{rgb}{0,0,0.545}


\lstset{
	language=c++,
	backgroundcolor=\color{white},
	basicstyle=\ttfamily\small,
	keywordstyle=\color{dblue}\bfseries,
	commentstyle=\color{cyan}\itshape,
	frame=lrtb,
	columns=fullflexible,
	showstringspaces=false,
	breaklines=true
}


% Info
\author{A. Sanfratello, A. Beconcini, F. Mola}
\title{Transformation from cartesian coordinates to polar coordinates using CORDIC}

\newcommand{\HRule}{\rule{\linewidth}{0.5mm}}

\begin{document}


\begin{titlepage}
\begin{center}
	\includegraphics[scale=.60]{img/Unipi_logo.jpg}\\[3cm]
	\textsc{\Large Digital System Design Project}
	\HRule \\[0.4cm]
{ \huge \bfseries Transformation from cartesian coordinates to polar coordinates using CORDIC \\[0.4cm] }
	\HRule \\[4cm]
	\noindent
	\begin{minipage}{0.4\textwidth}
	\begin{flushleft} \large
	\emph{Authors:}\\
	Alessio Sanfratello\\
	Andrea Beconcini\\
	Francesco Mola 
	\end{flushleft}
	\end{minipage}%
	\begin{minipage}{0.4\textwidth}
	\begin{flushright} \large
	\emph{Supervisor:} \\
	Prof. Luca Fanucci
	\end{flushright}
	\end{minipage}

	\vfill
	{\large Academic Year 2014/2015}
\end{center}
\end{titlepage}

\tableofcontents

\chapter{Introduction}

\section{The CORDIC algorithm}
\label{cordic_alghorithm}
The goal of this project is to design an integrated digital circuit which implements a converter from cartesian coordinates to polar ones, using the CORDIC algorithm.

CORDIC is an acronym for \emph{COordinate Rotation DIgital Computer} and it was first described by Jack E. Volder in 1959.

CORDIC has two mode of operation: \emph{rotation} and \emph{vector}. The former mode takes the coordinates of an input vector plus an angle of rotation and returns the new coordinates after the rotation has been applied.

The vector mode can convert an input vector from cartesian to polar coordinates and its result depends on multiple iterations of the CORDIC rotation mode. Basically, vector mode rotates the input vector until its \emph{y} coordinate became 0, so the modulus of the given vector is exactly equal to the value of the \emph{x} coordinate of the rotated vector and its angle is equal to the opposite of the total rotation angle.\newline
The rotation angle performed at $i$-th iteration $\alpha$ is

\begin{equation}
\alpha_{i} = \arctan \left(\dfrac{1}{2^i}\right)
\end{equation}

The reason for choosing such angles is that the rotated coordinates $x_{i+1}$ $y_{i+1}$  after $i$ rotation became

\begin{equation}
x_{i+1} = x_{i} - d_{i} \cdot y_{i} \cdot 2^{-i}
\end{equation}

\begin{equation}
y_{i+1} = y_{i} + d_{i} \cdot x_{i} \cdot 2^{-i}
\end{equation}

Where $d_{i}$ is equal to $+1$ if $y_{i} < 0$ and $-1$ otherwise.

The rotated coordinates can be computed using just sums and shift operation, as we can see in the previous equations.

The rotation performed at each stage $i$ is equal to $\alpha_{i}$ and the total rotation is given by the sum of the previous contribution such that the total rotation performed at iteration $i+1$ is

\begin{equation}
z_{i+1} = z_{i} + d_{i} \arctan \left(\dfrac{1}{2^i}\right)
\end{equation}


The results of the $\arctan()$ function can be stored in a ROM since this algorithm has to apply this function to a limited set of value. Such set depends on the number of iteration we are interested in.

Note that the CORDIC algorithm do not produce a correct output if input is equal to the null vector. In fact such input generates a null vector at every iteration while the total rotation angle keeps increasing. Such a behavior produces a wrong output.
 

The aim of this work is to produce a component which implements the CORDIC vector mode.

\section{Specification of requirements}

\begin{center}
\includegraphics[scale=.50]{img/cd.jpg}\\
\end{center}
Our network has five inputs and four outputs. \emph{x0} and \emph{y0} are the Cartesian coordinates which have to be transformed in polar coordinates. 

After entering the values of these inputs the user has to put \emph{soc} to the high value (\emph{1}) to let the network starting the conversion. To obtain the result the user has to put again the input \emph{soc} to the low value (\emph{0}); when the output \emph{eoc} goes to the high value the result can be read as modulus and angle respectively in the outputs \emph{rho} and \emph{theta}, the values obtained are given in fixed point with a notation Q8.24 (8 bits for the integer part, 24 bits for the decimal part). 

The value of the angle (\emph{theta}) is always correct and is given in radians, instead the value of the modulus (\emph{rho}) is correct if and  only if the output \emph{valid} is on the high value, otherwise it means that the value can't be represented on two's complement on the given number of bits. 

The input \emph{rst} is low triggered.


\chapter{Architecture}

\section{Input and output notation}
Our circuit deals with 32-bit inputs and outputs in Q8.24 fixed point notation. This means our network uses 8 bits for the integer part and 24 bits for the decimal part of its input and output data.

This notation let us represent value in the interval $ \left[ -128; 127.9999999 \right]$ and it gives us enough precision to represent $2^{-24}$ radians.

\section{ROM and addresses generator}
In order to compute the rotation angle at each iteration, we need the value of $\alpha_{i}$ for each possible iteration $i$.\\
As we said in \ref{cordic_alghorithm}, we use a 64x32bit ROM to store the arctan() values.

The ROM is filled such that the first 25 location contain angles $\alpha$ like

\begin{equation}
\alpha_{i} = \arctan \left( - \dfrac{1}{2^{i}} \right) \mbox{ with} \; i \in \left[0; 24 \right]
\end{equation}

The location from address 32 to 56 are filled with angles $\alpha$ like

\begin{equation}
\alpha_{i} = \arctan \left(\dfrac{1}{2^{i}} \right) \mbox{ with} \; i \in \left[0; 24 \right]
\end{equation}

We can not go any further storing the arctan() values because we do not have enough precision to represent them.

We choose to fill the remaining locations with zeros. Doing this, we do not modify the value of the output angle when the conversion is complete.

\section{Extender and is\textunderscore valid components}
\begin{center}
\includegraphics{img/isvalid.jpg}\\
\end{center}
The \emph{extender} is a component that simply extends the input of 2 bits, in particular this component extends the values of \emph{x0} and \emph{y0}. We compute these operations in order to ensure that the modulus can always be represented. In fact the maximum value that we can obtain for the modulus is when both \emph{x0} and \emph{y0} are identical and equal to the maximum and so the resultant modulus is:

		\begin{equation}
		x0 \cdot \sqrt{2} \cdot A_{n}
		\end{equation}
	\\	
where the last two values are constant and smaller than 2 and so their product is strictly smaller than 4. In this way with 2 bits given by the extender we are sure to represent the modulus. 

After the divider ( we remind that at the end of the $n^{th}$ iteration the value of $x_{n}$ represents the modulus times $A_{n}$ and so there is the necessity to divide it and moreover after the division we are sure that the modulus can be represented over n-1 bits), the modulus goes to the enter of the is \textunderscore valid block. 

This component waits until \emph{soc} is at the low value and when the algorithm is at the last iteration and after that computes the \emph{valid} output, simply checking the equality between the two most significant bits. After this operation the block puts on the output \emph{rho} the reduced modulus and puts the output \emph{eoc} to the high value, to notify the user that the conversion is over and the modulus is available.

\section{Counter}
Since our circuit performs multiple iterations to reach the expected output values, we need a counter to stop the computation.
We can stop the algorithm at the 25th iteration since our data format does not support precision provided with longer computation, in other words, the rotation angles after the 25th iteration are too small to be represented with Q8.24.

\section{soc, eoc, rst}
We provide the user some bit to control the working status of our cordic circuit. In particular, we provide the following
\begin{itemize}
	\item Soc (Start Of Conversion): This tells the circuit to take datas from the input bus ($x_{0}$ and $y_0$) and start the conversion.
	\item Eoc (End Of Conversion): this is an output bit set by the cordic circuit when the conversion is over and the user can read the result on the output bus (rho and theta);
	\item Rst (ReSeT): this input bit is low triggered and it must be set to 0 when we want to ensure cordic circuit to be in a consistent state, so it must be set to 0 every time the user want to perform the first conversion.
\end{itemize}

In order to drive the circuit correctly, the user must seto rst to 0 then, when the rst is set to 1 again, he can set input coordinates on the bus and notify the circuit via soc bit that data on $x_0 y_0$ are valid and the conversion can start.

\subsection{Starter}
Our circuit includes a specific submodule in order to implement the correct starting process. This Submodule is called \emph{Starter}. The Starter module check the value of the input bits soc and rst and provide a reset bit to all the cordic's registers. In particular it triggers a reset when the user sets rst to 0. Even when the user set the soc bit to 1, the reset goes to 0 for 10ns, so every register (including the iteration counter) is reset.

\pagebreak
\begin{figure}
\centering
\includegraphics[scale=0.52]{img/blockCordic.jpg}
\caption{General Overview\label{fig:block}}
\end{figure}


\chapter{Testing}

\section{Choice of testbench}
We designed two different kind of tests to check the correctness of both outputs rho and theta respectively.
The former ensure the modulus and validity bit while the former one validates the output angle.\\
To perform those we used a VHDL testbench file which can take an external input file containing input and expected values and compare cordic results with the expected ones.\\
The VHDL testbench and code for generating its input files can be found at the end of this document.

\begin{lstlisting}[caption={	A snippet from the testbench network}]
begin
	
		file_open(file_TEST, "test_cordic.txt", read_mode);
		rst <= '1';
		
		while not endfile(file_TEST) loop
			readline(file_TEST, v_ILINE);
			hread(v_ILINE, v_x0);	--reading first input
			hread(v_ILINE, v_y0);	--read second input
			hread(v_ILINE, v_theta);--reading expected theta
			hread(v_ILINE, v_rho);	--reading expected rho
			read(v_ILINE, v_valid);	--reading valid bit
			
			--wait for 10 ns;
			
			--wait for 60 ns;
			x0 <= v_x0;
			y0 <= v_y0;
			
			wait for 10 ns;
			soc <= '1' after 0 ns, '0' after 10 ns;
			wait for 10 ns;
			 
			while eoc = '0' loop
				wait for 10 ns;
			end loop;
			
			wait for 10 ns;
			
			write(v_OLINE, "rho ");
			hwrite(v_OLINE, rho);
			write(v_OLINE, " ");
			hwrite(v_OLINE, v_rho);
			write(v_OLINE, "  theta");
			write(v_OLINE, " ");
			hwrite(v_OLINE, zn);	   
			write(v_OLINE, " ");
			hwrite(v_OLINE, v_theta);
			write(v_OLINE, "  valid bit ");
			write(v_OLINE, valid);
			writeline(OUTPUT, v_OLINE);
			
			h_rho := rho(N-1 downto 16);
			h_theta := zn(N-1 downto 16);
			
			assert valid = v_valid
				report "Error on valid bit"
				severity ERROR;	  
				
			if v_valid = '1' then 
				assert h_rho = v_rho(N-1 downto 16)
					report "Error on rho"
					severity ERROR;
			end if;
				
			assert h_theta  = v_theta(N-1 downto 16)
				report "Error on theta"
				severity ERROR;	  
				
		end loop;
		
		file_close(file_TEST);
		write(v_OLINE,"End of test");	
		writeline(OUTPUT,v_OLINE);
		
		wait;
	end process;
\end{lstlisting}

\subsection{Testing modulus and validity bit}
In this test we feed our circuit with coordinates of points which belongs to the function
	\begin{equation}
		y = x
  	\end{equation}
We start from point $P = \left(-127, -127\right)$ and move toward the first quadrant with a step equal to 1 until we reach point $P=(127,127)$.\\
Then we concentrate our tests near the point which produce a \emph{rho} overflow. Such overflow occurs around $P=(90.5,90.5)$, so we concentrate our test around that point with a step equal to 0.01 and 0.001.

\subsection{Testing angle}
After having tested the modulus output (\emph{rho}), we check the output angle correctness. We start from a point whose polar coordinates are $\rho=127$ and $\theta=0$, then we move on a circumference with an angular step equal to $\frac{\pi}{24}$.

\section{Waveform Examples}
In the following figures you can see some examples of how signals evolve in time for a couple of different inputs.

In particular in figure \ref{fig:bisector} we pass $[1,1]$ as input and after 25 iterations we obtain congruent result.


\begin{figure}[!h]
\centering
\includegraphics[width=\textwidth]{img/test-bisettrice.png}
\caption{Waveform example of valid output\label{fig:bisector}}
\end{figure}

If instead we use $[120,126]$ as input, we obtain an invalid result, therefore valid flag has been set to 0.

\begin{figure}[!h]
\centering
\includegraphics[width=\textwidth]{img/test-valid.png}
\caption{Waveform example of invalid output\label{fig:invalid}}
\end{figure}






\chapter{Synthesis using Xilinx ISE Tool}
At the end we decide to synthesize our project using the Xilinx ISE Tool. We decide to set a balanced design goal which automatically sets the right process properties to achieve this optimization. After that we start our synthesis.
After fixing some warnings not shown by the simulation tool, we get this result shown in the Device Utilization Summary. 

\begin{center}
\includegraphics[scale=0.9]{img/utilizationSummary.png}
\end{center}


And this is the timing report with the calculation of the maximum path delay which gives us the relative time constraint that we have to respect in order to obtain the maximum frequency to use to drive our network.

\begin{center}
\includegraphics[scale=0.9]{img/timingReport.png}
\end{center}


\appendix

\chapter{Source code}

\section{Generation of test data using C++}

\begin{lstlisting}[caption={Test-data generation}]
#include <stdio.h>
#include <math.h>
#include <stdint.h>

#define TWO_TO_31 0x80000000
#define MUL 16777216
#define N_ENTRIES 32
#define DIV 24

struct Polar {
    int32_t rho;
    int32_t theta;
    bool valid;
};

int32_t rom[2*N_ENTRIES];

Polar cordic(int32_t x0, int32_t y0) {
    int64_t x = 0, y = 0;
    int32_t z = 0;
    
    // Extention to "34" bits (using int64_t)
    x = x0;
    y = y0;
    
    // Moving to first or fourth quadrant
    x = (x < 0) ? -x : x;
    
    
    for (int i=0; i<=25; i++) {
        int64_t x_old = x, y_old = y;
        int32_t z_old = z;
        
        // Computing d
        int d = (y < 0) ? 1 : -1;
        
        // Computing (inverter + mux + n_shift + adder)
        x = x_old - ((d * y_old) >> i);
        y = y_old + ((d * x_old) >> i);
        
        // reading  ROM entry
        int entry = (d > 0) ? i + N_ENTRIES : i;
        
        // Computing z using ROM's value
        z = z_old + rom[entry];
    }
    
    // Preparing result
    Polar result = {0,0,false};
    
    // Computing rho
    const int32_t A_n_inverted = 0x009B74ED;
    int64_t temp_rho = ((x >> 2) * A_n_inverted) >> 22;
    result.valid = (temp_rho >= TWO_TO_31) ? false : true;
    result.rho = (int32_t) temp_rho;
    
    // Computing theta
    const int32_t PI = 0x03243F6B;
    if (x0 < 0) {
        if (y0 >= 0) {
            result.theta = z + PI;
        } else {
            result.theta = z - PI;
        }
    } else {
        result.theta = -z;
    }
    
    return result;
}

void gen_rom() {
    // Computing negative values for atan(2^-i)
    for (int i = 0; i < N_ENTRIES; i++) {
        double a = (-1)*atan(pow(0.5, (double) i));
        rom[i] = (i < 25) ? (int32_t) (round(a*MUL)) : 0;
    }
    
    // Computing positive values for atan(2^-i)
    for (int i = 0; i < N_ENTRIES; i++) {
        double a = atan(pow(0.5, (double) i));
        rom[i+N_ENTRIES] = (int32_t) (round(a*MUL));
    }
}

int main() {
    // Filling ROM entries
    gen_rom();
    
    // Bisector test [-127,127]
    for (int k=-127; k<=127; k++) {
        Polar p = cordic(k<<24,k<<24);
        
        printf("%08X %08X %08X %08X %d\n", k<<24, k<<24, p.theta, p.rho, p.valid);
    }
    
    // Bisector test [90,91]
    for (int k=0; k<=100; k++) {
        double x = 90.0 + k/100.0;
        double y = 90.0 + k/100.0;
        
        int32_t x_int = (int32_t) (round(x*MUL));
        int32_t y_int = (int32_t) (round(y*MUL));
        
        Polar p = cordic(x_int,y_int);
        
        printf("%08X %08X %08X %08X %d\n", x_int, y_int, p.theta, p.rho, p.valid);
    }
    
    // Bisector test [90.49,90.52]
    for (int k=0; k<=30; k++) {
        double x = 90.49 + k/1000.0;
        double y = 90.49 + k/1000.0;
        
        int32_t x_int = (int32_t) (round(x*MUL));
        int32_t y_int = (int32_t) (round(y*MUL));
        
        Polar p = cordic(x_int,y_int);
        
        printf("%08X %08X %08X %08X %d\n", x_int, y_int, p.theta, p.rho, p.valid);
    }
    
    // Angles test
    const double r = 127;
    for (int k=0; k<2*DIV; k++) {
        double a = k * M_PI / DIV;
        int32_t x_int = round(r*cos(a)*MUL);
        int32_t y_int = round(r*sin(a)*MUL);
        
        Polar p = cordic(x_int,y_int);
        
        printf("%08X %08X %08X %08X %d \n", x_int, y_int, p.theta, p.rho, p.valid);
    }
    
    return 0;
}

\end{lstlisting}

\end{document}
